O Raspberry foi programado para atuar como duas interfaces principais. Como access point da rede wireless e como broker do protocolo MQTT. As configurações utilizadas para tal função foram:

\begin{enumerate}
    \item Atualização e instalação dos pacotes:
        \begin{itemize}
            \item \$ sudo apt-get update
            \item \$ sudo apt-get install isc-dhcp-server
            \item \$ sudo apt-get install hostapd
            \item \$ sudo apt-get install mosquitto
            \item \$ sudo apt-get install mosquitto-dev
        \end{itemize}
    
    \item No arquivo \textit{/etc/dhcp/dhcp.conf} fazer as seguintes alterações:
        \begin{itemize}
            \item Comentar com # as linhas:
                option domain-name "example.org";
                
                option domain-name-servers 
                
                ns1.example.org, ns2.example.org
            \item Descomentar as linhas:
                authoritative
            \item Escrever no fim do arquivo:

                subnet 10.0.0.0 netmask 255.255.255.0 $\{$

                \hspace*{6mm}    range 10.0.0.2 10.0.0.255;
                
                \hspace*{6mm}    option domain-name-servers 8.8.8.8;
                
                \hspace*{6mm}    option domain-name "local";
                
                \hspace*{6mm}    option routers 10.0.0.1;
                
                \hspace*{6mm}    option broadcast-address 10.0.0.255;
                
                \hspace*{6mm}    default-lease-time 600;
                
                \hspace*{6mm}    max-lease-time 7200;
                $\}$
        \end{itemize}
    \item No arquivo \textit{/etc/network/interfaces} colocar no arquivo:
        auto lo
        
        iface lo inet loopback
        
        iface eth0 inet dhcp
        
        allow-hotplug wlan0
        
        iface wlan0 inet static
        
            \hspace*{6mm} address 10.0.0.1
        
            \hspace*{6mm} netmask 255.255.255.0
        
	    \hspace*{6mm} gateway 10.0.0.1

    \item Criar o arquivo \textit{/etc/hostapd/hostapd.conf} com o seguinte texto:

        interface=wlan0

        driver=nl80211

        ssid=bikerift\_broker

        country\_code=US

        hw\_mode=g

        channel=6

        macaddr\_acl=0

        auth\_algs=1

        ignore\_broadcast\_ssid=0

        wpa=2

        wpa\_passphrase=bikerift 

        wpa\_key\_mgmt=WPA-PSK

        wpa\_pairwise=CCMP

        wpa\_group\_rekey=86400

        ieee80211n=1

        wme\_enabled=1

    \item Alterar no arquivo \textit{/etc/default/hostapd} a linha #DAEMON\_CONF 

por DAEMON\_CONF="/etc/hostapd/hostapd.conf"

    \item No arquivo \textit{/etc/sysctl.conf} descomentar a linha: net.ipv4.ip\_forward=1.

    \item Executar no terminal:
        \begin{itemize}
            \item \$ sudo sh -c "echo 1 > /proc/sys/net/ipv4/ip\_forward"
   
            \item \$ sudo iptables -t nat -A POSTROUTING -o eth0 -j MASQUERADE
   
            \item \$ sudo iptables -A FORWARD -i eth0 -o wlan0 -m state --state RELATED,ESTABLISHED -j ACCEPT

            \item \$ sudo iptables -A FORWARD -i wlan0 -o eth0 -j ACCEPT 

            \item \$ sudo sh -c "iptables-save > /etc/iptables/rules.v4"

            \item \$ sudo service hostapd start

            \item \$ sudo service isc-dhcp-server start

            \item \$ sudo update-rc.d hostapd enable

            \item \$ sudo update-rc.d isc-dhcp-server enable

        \end{itemize}
\end{enumerate}