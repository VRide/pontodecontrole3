\begin{apendicesenv}

\partapendices

\chapter{Termo de Abertura do Projeto}

\subsection{Descrição do projeto}

O projeto é uma plataforma de ciclismo interativa com imersão em ambiente de realidade virtual e monitoramento de dados fisiológicos e de desempenho. O usuário utilizará uma bicicleta acoplada ao sistema e um óculos de realidade virtual, interagindo com um ambiente virtual gamificado emulado pelo Oculus por meio da bicicleta, pedalando e utilizando o guidão. 

\subsection{Justificativa do projeto}

Treinos de ciclismo em ambientes fechados é uma demanda relevante para profissionais e entusiastas do esporte. Diversas soluções para esse ponto existem, incluindo circuitos \textit{indoor} e bicicletas fixas de exercício que simulam parte da estrutura de uma bicicleta tradicional. Tais soluções são o padrão atual do mercado, mas possuem algumas limitações. Uma delas é o fato de que bicicletas de exercício não possuem uma sensação similar o suficiente a uma bicicleta comum, além de possuírem indicadores limitados a quantidade aproximada de calorias gastas e outros. Treinamento em circuitos \textit{indoor} requerem deslocamento do usuário ao local.

Um sistema de ciclismo interativo permitiria uma experiência de ciclismo verossímil e gamificada, servindo de estímulo de treinamento aos usuários, tanto pela qualidade da experiência quanto pela praticidade de uso.


\subsection{Objetivos do projeto}

Como proposta de solução, o projeto a ser desenvolvido tem como objetivo emular a sensação de andar de bicicleta de uma forma verossímil, utilizando uma bicicleta tradicional como base e a realidade virtual para emular diferentes ambientes de ciclismo, a fim de aproximar a experiência do usuário no sistema à experiência real, à medida em que dados do seu desempenho são coletados. O projeto também procura ser de fácil acesso e utilização, utilizando-se de diretrizes de usabilidade para isso. Com isso, busca-se uma maior utilização do sistema no dia-a-dia, para que o  usuário tenha seu desempenho medido de forma constante.

\subsection{Requisitos de alto nível}

O projeto terá uma bicicleta acoplada a uma plataforma, um óculos de realidade virtual e um \textit{notebook} para a execução do \textit{software}. O sistema também contará com um sistema que permite a elevação da bicicleta e aumento no esforço necessário para movimentar o pedal, simulando subidas e decidas.

O jogo contará com dois circuitos, um com pista lisa e duas curvas inspirado em pistas tradicionais e um com caminho com elevações e diversas curvas, inspirado em ambientes abertos. Cada usuário poderá ter uma conta local, em que serão registrados preferências e dados de desempenho. Os dados de desempenho devem ser apresentados em forma de gráficos para o usuário e acessíveis a partir do menu principal. A bicicleta deverá ser projetada de tal forma que caso o usuário venha a perder o equilibrio este não caia da plataforma. Além disso, deverá haver aproveitamento de parte da energia produzida pelo usuário ao pedalar.
 
\subsection{Subsistemas Identificados}

Foram identificados quatro subsistemas no projeto: o subsistema eletrônico, responsável pela construção dos componentes eletrônicos, captação e interpretação dos sinais da bicicleta e do usuário, o subsistema de energia, responsável pela geração de energia elétrica a partir da energia mecânica obtida pelos movimentos de pedalada do usuário, o subsistema de estrutura, responsável pela construção da plataforma e adaptação da bicicleta e componentes do Oculus na montagem do projeto, e o subsistema de software, responsável pela interface de interação com o usuário e apresentação do cenário virtual.

\subsection{Riscos}

Os principais riscos que podem ocorrer durante o projeto são os referentes a inexperiência da equipe na utilização das ferramentas necessárias para construção da plataforma, bem como nas tecnologias a serem utilizadas no desenvolvimento. Outros riscos que podem ocorrer e que afetam diretamente ao projeto, são os riscos relacionados a utilização de equipamentos de alto custo e do espaço cedido para equipe no LART. Outro risco de grande impacto é a integração no último mês do projeto de todos os módulos a serem desenvolvidos.

\subsection{Resumo do cronograma de marcos}

O desenvolvimento da solução terá três marcos principais, caracterizados como pontos de controle. A primeira fase do projeto é a fase de planejamento e tem duração de três semanas e será finalizada no dia do primeiro ponto de controle, entre seis a nove de setembro. A segunda fase é a fase de construção dos subsistemas e inicia dia primeiro de setembro e termina dia três de novembro, com uma duração de, aproximadamente, dois meses. O segundo ponto de controle ocorre entre os dias primeiro e três de novembro. A terceira fase é a de integração dos subsistemas, inicia dia três de novembro e termina dia primeiro de dezembro. O terceiro ponto de controle ocorre entre três a seis de dezembro.

\subsection{Resumo do orçamento}

O custo total do projeto será de R\$ 8.995,34, deste valor total, R\$ 7.317,64, refere-se aos custos do óculos \textit{Rift} e o Desktop do LART que a equipe já possui.

\begin{table}[htp]
\centering
\caption{Orçamento}
\label{orcamento}
\begin{tabular}{|l|l|}
\hline
Simulador de Ambiente Virtual & R\$ 7.317,64 \\ \hline
Sistema de Alimentação de Energia & R\$ 383,99 \\ \hline
Estrutura & R\$ 900,00 \\ \hline
Sistemas de aquisição e controle & R\$ 393,71 \\ \hline
\multicolumn{2}{|c|}{\textbf{R\$ 8.995,34}} \\ \hline
\end{tabular}
\end{table}

\subsection{Lista das partes interessadas}

Dentre os interessados, estão os patrocinadores e alunos de graduação de engenharia da Universidade de Brasília campus Gama que cursam a disciplina de Projeto Integrador em Engenharia 2.

\subsection{Requisitos para a aprovação do projeto}

O projeto deverá ser aprovado por seus patrocinadores e professores da disciplina de Projeto Integrador em Engenha2.

\subsection{Gerência do projeto}

O projeto será gerenciado por Matheus Pereira Santana, graduando em Engenharia Eletrônica, sendo ele o gerente geral. Mas terá também outros quatro estudantes responsavéis por gerenciar cada subsistema da Plataforma de Ciclismo Interativa.

\subsection{Patrocinadores}

O projeto contará com dois patrocinadores, a professora Carla Rocha e o professor Augusto Brasil, que possuem interesse no produto a ser gerado ao final do semestre. 

\end{apendicesenv}
